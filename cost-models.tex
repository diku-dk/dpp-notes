\chapter{Cost Models}

When analysing the performance of algorithms, we do so not by
implementing them on some concrete machine and measuring their
performance, but rather by mathematically quantifying how many
``steps'' they need to finish. In your previous studies, you may have
used an intuitive notion of what a ``step'' is, such as counting
program statements or arithmetic operations. When analysing parallel
algorithms, we need more precision, since we are not just interested
in the total amount of work, but also the nature of the work---i.e.,
whether it is parallel.

\begin{definition}[Cost model]
  A mathematical model used for algorithmic analysis that defines the
  cost of various operations in some specified language. May contain
  one or more \emph{cost measures}.
\end{definition}

\begin{definition}[Cost measure]
  A (usually) discrete measure for the cost of a computation along
  some axis, such as sequential work, total work, or space usage.
\end{definition}

\begin{definition}[Work]
  The total amount of steps in an execution.
\end{definition}

\begin{definition}[Span/depth]
  The length of the longest sequential dependency chain in an
  execution.
\end{definition}


%%% Local Variables:
%%% mode: LaTeX
%%% TeX-master: "dpp-notes"
%%% End:
