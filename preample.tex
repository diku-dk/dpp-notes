% General stuff.  The idea is that this preample can also be used to
% compile each chapter separately.

\usepackage[utf8]{inputenc}

\usepackage{ntheorem}
\usepackage{amsmath}[ntheorem]
\usepackage{mathtools}
\usepackage{amsfonts}
\usepackage{thmtools}

% General notation macros that may also be useful when authoring
% slides or other material.


\usepackage[table]{xcolor}
\usepackage{caption}
\usepackage{subcaption}
\usepackage{graphicx}
\usepackage{todonotes}
\usepackage{url}
\usepackage{hyperref}

\usepackage{cleveref}

\usepackage{couriers}
\usepackage{listings}
\lstset{basicstyle=\ttfamily}
% A listings language definition for Futhark.

\lstdefinelanguage{futhark}
{
  % list of keywords
  morekeywords={
    do,
    else,
    for,
    if,
    in,
    include,
    let,
    loop,
    then,
    type,
    val,
    while,
    with,
    module,
    def,
    entry,
    local,
    open,
    import,
    assert,
    match,
    case,
  },
  sensitive=true, % Keywords are case sensitive.
  morecomment=[l]{--}, % l is for line comment.
  morestring=[b]" % Strings are enclosed in double quotes.
}


% Teach cleveref about listings.
\crefname{lstlisting}{listing}{listings}
\Crefname{lstlisting}{Listing}{Listings}

\newtheorem{definition}{Definition}[chapter]
\newtheorem{theorem}{Theorem}[chapter]
\newtheorem{corollary}[theorem]{Corollary}
\newtheorem{example}{Example}[chapter]

% Tikz stuff
\usepackage{tikz}
\usetikzlibrary{calc}
\usetikzlibrary{shapes.multipart}
\usetikzlibrary{shapes.arrows}
\usetikzlibrary{chains}
\usetikzlibrary{arrows}
\usetikzlibrary{matrix}
\usetikzlibrary{arrows.meta}
\usetikzlibrary{intersections}
\usetikzlibrary{cd}
\usetikzlibrary{fit}
\usetikzlibrary{backgrounds}
\usetikzlibrary{decorations.pathreplacing}

\setsecnumdepth{subsubsection}

%%% Local Variables:
%%% mode: latex
%%% TeX-master: "notes"
%%% End:
