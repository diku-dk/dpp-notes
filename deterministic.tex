\chapter{Deterministic Parallel Programming}

Modern computers have become ever more parallel, and this trend will
continue in the future. The reasons are ultimately related to
fundamental physical constraints---the speed of information is limited
by the speed of light (and the speed of electrons in silicon is lower
still), and transistors cannot change their state more than ever so
often. Instead of increasing the clock frequency, processor designers
therefore increase parallelism, and today even single-chip processors
may have hundreds of cores.

Further, often called \emph{accelerators}, have also become popular
because they tend to provide more computational throughput for a given
power or transistor budget.

\begin{definition}[Accelerator]
  Computer hardware designed to efficiently perform a specific task,
  typically more efficiently than general-purpose hardware.
\end{definition}

Some accelerators are extremely limited, and perform only completely
fixed tasks, such as decoding network packets or implementing
cryptographical operations. Other accelerators, such as most GPUs, are
fully programmable strictly speaking \emph{general purpose}, in that
they can perform any computation. However, they perform best on a
particular subset of problems---in particular those that have ample
parallelism and a predictable memory access patterns.

One challenge of massively parallel computing is that while the human
mind is good at reasoning about things that occur in sequential order,
it is bad at reasoning about concurrency. It is telling that the field
of concurrent programming has such a rich collection of terms for how
things can go wrong: race condition, deadlock, livelock, starvation,
priority inversion, et cetera. The key problem is that such programs
are often nondeterministic, with the outcome of the program depending
on complex and unpredictable factors about how the computation is
scheduled.

It is unlikely that we will be able to significantly improve the human
mind in the near future---indeed, it is little changed from when our
ancestors chased animals on the tundras---so instead we will have to
find ways to adapt our modes of programming to what we as humans can
handle. It is important here to distinguish the notions of concurrency
and parallelism. These terms have often been used interchangeably, but
for the purposes of these notes, we will use the following definitions
(which also largely match modern usage in academia and industry).

\begin{definition}[Concurrency]
  The use of multiple, possibly interacting, logical flows of control.
\end{definition}

\begin{definition}[Parallelism]
  The simultaneous use of multiple processing units, with the goal of
  speeding up a computation.
\end{definition}

Parallelism is what we desire, as it allows more efficient use of
hardware, but we want to avoid the complexities of concurrency.
Fortunately, high level programming models exist that enable
deterministic (or nearly deterministic) parallel programming.

\begin{definition}[Deterministic program]
  A program that for the same input always produces the same output.
  In particular, it is not sensitive to how (or if) it is executed in
  parallel.
\end{definition}

The specific kind of parallelism we will study in DPP is data
parallelism.

\begin{definition}[Task parallelism]
  Simultaneously performing \textit{different} operations on potentially
  different pieces of data.  Often nondeterministic.
\end{definition}

\begin{definition}[Data parallelism]
  Simultaneously performing \textit{the same} operation on different
  pieces of the same data.  Almost always deterministic.
\end{definition}

Data parallelism as a programming model is interesting because it
mixes human and machine sympathy - meaning they are easy to reason
about and easy to execute. The downside is that not all problems are
natural to express in a data parallel setting.

\begin{definition}[Probstring's Law]
  Compiler advances double computing power every 18 years.
\end{definition}

%%% Local Variables:
%%% mode: LaTeX
%%% TeX-master: "dpp-notes"
%%% End:
